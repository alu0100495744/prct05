\documentclass{beamer}
\usepackage[spanish]{babel}
\usepackage[utf8]{inputenc}
\usepackage{graphicx}

\usecolortheme[RGB={122,59,122}]{structure}

\newtheorem{definicion}{Definición}
\newtheorem{ejemplo}{Ejemplo}

\title{Manejo básico de beamer}
\author{Javier Hernández Pérez}
\date{\today}
\usetheme{Madrid}
\begin{document}
\begin{frame}
\titlepage
\end{frame}
%++++++++++++++++++++++++++++++++++++++++++++++++++++++++++++++++++++++++++++++++++++
%++++++++++++++++++++++++++++++++++++++++++++++++++++++++++++++++++++++++++++++++++++
\begin{frame}
  \frametitle{Contenido}
  \tableofcontents
\end{frame} 
%++++++++++++++++++++++++++++++++++++++++++++++++++++++++++++++++++++++++++++++++++++
%++++++++++++++++++++++++++++++++++++++++++++++++++++++++++++++++++++++++++++++++++++
\section{Motivación y objetivos}
\begin{frame}
\frametitle{Motivación}
Ante la entrada de los ordenadores cada día es más importante ser capaces de hacer una presentación correctamente para:
\pause
\begin{enumerate}[<+->]
\item Explicar un trabajo.
%\pause
\item Vender algo.
%\pause
\item No ser llamado anticuado.
\pause
\end{enumerate}
Beamer que es software libre, permite hacer sin pagar presentaciones de gran calidad.
\end{frame}
%++++++++++++++++++++++++++++++++++++++++++++++++++++++++++++++++++++++++++++++++++++
%++++++++++++++++++++++++++++++++++++++++++++++++++++++\usecolortheme{albatross}++++++++++++++++++++++++++++++
\section{Como hacer una bonita presentación}

\begin{frame}
  \frametitle{Presentación} 

    En una presentación no debe ponerse toda la información sino una guía que debe:
    \pause

    \begin{itemize}
    \item
    Fijar conceptos a los oyentes.
    \pause

    \item
    Ser concisa.

  \end{itemize}

\end{frame}

%++++++++++++++++++++++++++++++++++++++++++++++++++++++++++++++++++++++++++++++++++++
%++++++++++++++++++++++++++++++++++++++++++++++++++++++++++++++++++++++++++++++++++++
\section{Fórmulas} 

\begin{frame}
  \frametitle{Formulas} 

\begin{example}
\begin{enumerate}
\item $ \{\emptyset\} \neq \mathbb{R} $ , pues $ 0 \in  \mathbb{R}$
\item $\dfrac {\partial f} {\partial x}=1 \Rightarrow f=x+c$ con c una constante.
\item $\int _{0}^{0} x dx =0$
\item \begin{displaymath}
       1+1=2
       \end{displaymath}
\item Si $n=2k+1$ con $ k\in \mathbb{N}$, n es impar.
\end{enumerate}

\end{example}


\end{frame}
%++++++++++++++++++++++++++++++++++++++++++++++++++++++++++++++++++++++++++++++++++++

%++++++++++++++++++++++++++++++++++++++++++++++++++++++++++++++++++++++++++++++ 
\section{Bibliografía}

\begin{frame}
  \frametitle{Bibliografía}

  \begin{thebibliography}{10}



    \beamertemplatebookbibitems
    \bibitem[URL: vision objects]{vo} 
    vision objects consultada 15-3-13 {\small \url{http://webdemo.visionobjects.com/equation.html?locale=default}}
    
    \beamertemplatebookbibitems
    \bibitem[URL: zweitag]{z} 
     zweitag consultada 15-3-13 {\small \url{http://detexify.kirelabs.org/classify.html}}
     
     \beamertemplatebookbibitems
    \bibitem[URL: The University of Sydney]{tuos} 
     The University of Sydney consultada 15-3-13 {\small \url{http://ci2.es/objetos-de-aprendizaje/elaborando-referencias-bibliograficas}}
     
  \end{thebibliography}
\end{frame}


%++++++++++++++++++++++++++++++++++++++++++++++++++++++++++++++++++++++++++++++  

\end{document}
